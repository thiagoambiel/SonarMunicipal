% ======================================================================
% Relatorio Cientifico (SBC-style) — Sonar Municipal / Sonar Digital
% Arquivo unico e auto-contido (inclui .bib via filecontents)
% Compilacao sugerida: pdflatex -> biber -> pdflatex -> pdflatex
% ======================================================================

\begin{filecontents*}{sonar-references.bib}
@online{interlegis_sapl,
  author   = {{Senado Federal} and {Programa Interlegis}},
  title    = {Sistema de Apoio ao Processo Legislativo (SAPL)},
  year     = {2026},
  url      = {https://www12.senado.leg.br/interlegis/produtos/sapl},
  urldate  = {2026-01-07}
}

@online{interlegis_sapl_repo,
  author   = {{Interlegis}},
  title    = {SAPL: Sistema de Apoio ao Processo Legislativo (repositório oficial)},
  year     = {2026},
  url      = {https://github.com/interlegis/sapl},
  urldate  = {2026-01-07}
}

@online{ibge_localidades_api,
  author   = {{IBGE}},
  title    = {API de Localidades (documentação)},
  year     = {2026},
  url      = {https://servicodados.ibge.gov.br/api/docs/localidades},
  urldate  = {2026-01-07}
}

@phdthesis{fielding_rest_2000,
  author = {Fielding, Roy Thomas},
  title  = {Architectural Styles and the Design of Network-based Software Architectures},
  school = {University of California, Irvine},
  year   = {2000},
  url    = {https://www.ics.uci.edu/~fielding/pubs/dissertation/top.htm},
  urldate = {2026-01-07}
}

@online{openapi_3_2,
  author   = {{OpenAPI Initiative}},
  title    = {OpenAPI Specification (v3.2.0)},
  year     = {2024},
  url      = {https://spec.openapis.org/oas/v3.2.0.html},
  urldate  = {2026-01-07}
}

@online{swagger_ui,
  author   = {{Swagger}},
  title    = {Swagger UI: REST API Documentation Tool},
  year     = {2026},
  url      = {https://swagger.io/tools/swagger-ui/},
  urldate  = {2026-01-07}
}

@online{drf_pagination,
  author   = {{Django REST framework}},
  title    = {Pagination},
  year     = {2026},
  url      = {https://www.django-rest-framework.org/api-guide/pagination/},
  urldate  = {2026-01-07}
}

@online{w3c_prov_dm,
  author   = {{W3C}},
  title    = {PROV-DM: The PROV Data Model},
  year     = {2013},
  url      = {https://www.w3.org/TR/prov-dm/},
  urldate  = {2026-01-07}
}

@article{wilkinson_fair_2016,
  author  = {Wilkinson, Mark D. and Dumontier, Michel and Aalbersberg, IJsbrand Jan and others},
  title   = {The FAIR Guiding Principles for scientific data management and stewardship},
  journal = {Scientific Data},
  year    = {2016},
  volume  = {3},
  pages   = {160018},
  doi     = {10.1038/sdata.2016.18}
}

@article{wang_e5_2024,
  author  = {Wang, Liang and Yang, Nan and Huang, Xiaolong and Yang, Linjun and Majumder, Rangan and Wei, Furu},
  title   = {Multilingual E5 Text Embeddings: A Technical Report},
  journal = {arXiv},
  year    = {2024},
  url     = {https://arxiv.org/abs/2402.05672},
  urldate = {2026-01-07}
}

@article{piau_ptt5v2_2024,
  author  = {Piau, Marcos and Vieira, Matheus P. and others},
  title   = {ptt5-v2: A Closer Look at Continued Pretraining of T5 Models for the Portuguese Language},
  journal = {arXiv},
  year    = {2024},
  url     = {https://arxiv.org/abs/2406.10806},
  urldate = {2026-01-07}
}

@article{dettmers_qlora_2023,
  author  = {Dettmers, Tim and Pagnoni, Artidoro and Holtzman, Ari and Zettlemoyer, Luke},
  title   = {QLoRA: Efficient Finetuning of Quantized LLMs},
  journal = {arXiv},
  year    = {2023},
  url     = {https://arxiv.org/abs/2305.14314},
  urldate = {2026-01-07}
}

@article{zhang_bertscore_2019,
  author  = {Zhang, Tianyi and Kishore, Varsha and Wu, Felix and Weinberger, Kilian Q. and Artzi, Yoav},
  title   = {BERTScore: Evaluating Text Generation with BERT},
  journal = {arXiv},
  year    = {2019},
  url     = {https://arxiv.org/abs/1904.09675},
  urldate = {2026-01-07}
}

@online{qdrant_overview,
  author   = {{Qdrant}},
  title    = {Qdrant Documentation},
  year     = {2026},
  url      = {https://qdrant.tech/documentation/},
  urldate  = {2026-01-07}
}

@online{qdrant_collections,
  author   = {{Qdrant}},
  title    = {Collections (conceitos)},
  year     = {2026},
  url      = {https://qdrant.tech/documentation/concepts/collections/},
  urldate  = {2026-01-07}
}

@online{hf_tei,
  author   = {{Hugging Face}},
  title    = {Text Embeddings Inference (TEI)},
  year     = {2026},
  url      = {https://huggingface.github.io/text-embeddings-inference/},
  urldate  = {2026-01-07}
}

@online{nextjs_docs,
  author   = {{Vercel}},
  title    = {Next.js Documentation},
  year     = {2026},
  url      = {https://nextjs.org/docs},
  urldate  = {2026-01-07}
}

@online{vercel_nextjs,
  author   = {{Vercel}},
  title    = {Next.js on Vercel (documentação)},
  year     = {2025},
  url      = {https://vercel.com/docs/frameworks/full-stack/nextjs},
  urldate  = {2026-01-07}
}

@online{openai_gpt51_model,
  author   = {{OpenAI}},
  title    = {GPT-5.1 Model (OpenAI API)},
  year     = {2026},
  url      = {https://platform.openai.com/docs/models/gpt-5.1},
  urldate  = {2026-01-07}
}

@online{openai_gpt51_system_card,
  author   = {{OpenAI}},
  title    = {GPT-5.1 Instant and GPT-5.1 Thinking: System Card Addendum},
  year     = {2025},
  url      = {https://openai.com/index/gpt-5-system-card-addendum-gpt-5-1/},
  urldate  = {2026-01-07}
}

@online{inep_microdados_censo_escolar,
  author   = {{INEP}},
  title    = {Censo Escolar: Microdados (dados abertos)},
  year     = {2026},
  url      = {https://www.gov.br/inep/pt-br/acesso-a-informacao/dados-abertos/microdados/censo-escolar},
  urldate  = {2026-01-07}
}
\end{filecontents*}

\documentclass[10pt,a4paper]{article}

% -------------------------
% Pacotes
% -------------------------
\usepackage[a4paper,margin=2.2cm]{geometry}
\usepackage[T1]{fontenc}
\usepackage[utf8]{inputenc}
\usepackage[brazil,english]{babel}
\usepackage{times}
\usepackage{microtype}
\usepackage[hidelinks]{hyperref}
\usepackage{url}
\usepackage{graphicx}
\usepackage{booktabs}
\usepackage{amsmath,amssymb}
\usepackage{enumitem}
\usepackage{caption}
\usepackage{float}
\usepackage{multirow}
\usepackage{siunitx}

% Biblatex ABNT
\usepackage{csquotes}
\usepackage[backend=biber,style=abnt,language=brazil,sorting=nty]{biblatex}
\addbibresource{sonar-references.bib}

% TikZ (corrigido: positioning)
\usepackage{tikz}
\usetikzlibrary{positioning,arrows.meta,calc,shapes}

% Algoritmos (corrigido: ambiente algorithmic)
\usepackage{algorithm}
\usepackage{algpseudocode}
\algrenewcommand\algorithmicrequire{\textbf{Entrada:}}
\algrenewcommand\algorithmicensure{\textbf{Saída:}}

% -------------------------
% Ajustes de estilo
% -------------------------
\setlength{\parindent}{0pt}
\setlength{\parskip}{6pt}
\captionsetup{font=small,labelfont=bf}

\newenvironment{resumo}{
  \selectlanguage{brazil}
  \begin{center}\textbf{Resumo}\end{center}
  \small
}{
  \normalsize
  \selectlanguage{brazil}
}
\newcommand{\keywords}[1]{\vspace{2pt}\noindent\textbf{Keywords:} #1}
\newcommand{\palavrasChave}[1]{\vspace{2pt}\noindent\textbf{Palavras-chave:} #1}

% -------------------------
% Metadados editáveis
% -------------------------
\title{Sonar Municipal: Recomendação de Políticas Públicas a partir de Projetos de Lei Municipais e Indicadores Oficiais}

\author{
  Autor(a) 1\thanks{Edite nomes, e-mails e afiliações conforme necessário.} \\
  Instituição \\
  \texttt{email@exemplo.com}
  \and
  Autor(a) 2 \\
  Instituição \\
  \texttt{email@exemplo.com}
}

\date{}

% ======================================================================
\begin{document}
\selectlanguage{brazil}

\twocolumn[
\maketitle

\selectlanguage{english}
\begin{abstract}
Municipal governments often need to draft, compare, and justify legislative proposals under limited time and resources. This report describes a reproducible method to discover Brazilian SAPL instances, extract municipal Bills (Projetos de Lei), transform their short summaries (ementas) into action-oriented recommendations, and retrieve relevant actions via semantic search. We also describe how time-series indicators from official sources can be used to approximate real-world effects, and how similar proposals can be grouped into policy clusters to improve statistical robustness. Finally, we present a web platform that operationalizes the method for non-technical users.
\keywords{municipal legislation; SAPL; semantic search; text-to-text models; open data; reproducibility}
\end{abstract}

\selectlanguage{brazil}
\begin{resumo}
Gestores municipais frequentemente precisam redigir, comparar e justificar proposições legislativas com restrição de tempo e de recursos. Este relatório descreve um método reprodutível para (i) descobrir instâncias do SAPL no Brasil, (ii) extrair Projetos de Lei (PLs), (iii) transformar ementas em recomendações de ação, e (iv) recuperar ações relevantes por busca semântica. Também descrevemos como indicadores em séries temporais de fontes oficiais podem ser usados para estimar efeitos no mundo real, e como agrupar projetos similares em ``políticas públicas'' aumenta o poder estatístico e a explicabilidade. Por fim, apresentamos uma plataforma web que torna o método acessível a usuários sem perfil técnico.
\palavrasChave{legislação municipal; SAPL; busca semântica; modelos texto-para-texto; dados abertos; reprodutibilidade}
\end{resumo}
\vspace{6pt}
]

% ======================================================================
\section{Introdução}

A produção legislativa municipal no Brasil ocorre em um cenário de alta diversidade institucional. Mesmo quando diferentes municípios enfrentam problemas semelhantes (por exemplo, segurança pública ou evasão escolar), as soluções propostas podem variar em linguagem, escopo e grau de detalhamento. Isso dificulta a reutilização de conhecimento legislativo entre municípios e reduz a eficiência da tomada de decisão baseada em evidências.

Este trabalho descreve o projeto \textbf{Sonar Municipal} (plataforma web: \url{https://sonar-municipal.vercel.app/}), cujo objetivo é apoiar a elaboração e análise de Projetos de Lei por meio de:
\begin{itemize}[leftmargin=*,nosep]
  \item descoberta e coleta sistemática de PLs em instâncias do SAPL;
  \item tradução de ementas para uma linguagem de ação recomendada;
  \item busca semântica de ações a partir da pergunta do usuário;
  \item simulação aproximada de efeitos em indicadores oficiais ao longo do tempo;
  \item agrupamento de PLs similares em ``políticas públicas'' para análise conjunta.
\end{itemize}

\textbf{Contribuições.} (C1) Método auditável para descoberta de instâncias SAPL e extração exaustiva de PLs via API. (C2) Tradução de ementas para ações recomendáveis. (C3) Busca semântica por embeddings. (C4) Análise com indicadores e agrupamento. (C5) Plataforma web para usuários não técnicos.

% ======================================================================
\section{Fundamentação e conceitos utilizados}

\subsection{SAPL e padronização do processo legislativo}

O \textbf{SAPL} é um sistema mantido pelo Programa Interlegis e amplamente adotado por casas legislativas brasileiras para informatizar e dar publicidade ao processo legislativo \cite{interlegis_sapl}. Por ser de código aberto \cite{interlegis_sapl_repo}, muitas instâncias compartilham estruturas semelhantes de consulta e, em vários casos, expõem uma API.

\subsection{APIs HTTP, REST, OpenAPI e Swagger}

Uma \textbf{API} (Interface de Programação de Aplicações) é um conjunto de regras que permite que programas acessem funcionalidades e dados de um serviço. Neste trabalho, consideramos APIs acessadas por HTTP.

O estilo \textbf{REST} (Representational State Transfer) organiza a API em recursos acessados por URLs, com operações padronizadas (por exemplo, GET para consulta) \cite{fielding_rest_2000}. Um \textbf{endpoint} é a URL associada a um recurso específico (por exemplo, uma lista de matérias).

\textbf{OpenAPI} descreve formalmente endpoints, parâmetros e respostas \cite{openapi_3_2}. O \textbf{Swagger UI} renderiza essa descrição em uma interface navegável para exploração e teste \cite{swagger_ui}.

\subsection{Paginação}

APIs costumam retornar coleções de forma \textbf{paginada}, dividindo resultados em páginas para reduzir carga e tamanho da resposta. Paginação é essencial para extração exaustiva de bases volumosas \cite{drf_pagination}.

\subsection{Embeddings, busca semântica e banco vetorial}

\textbf{Embedding} é um vetor numérico que representa um texto de modo que textos semanticamente próximos fiquem próximos no espaço vetorial. \textbf{Busca semântica} recupera itens por similaridade entre embeddings, e não por coincidência literal.

Um \textbf{banco vetorial} armazena vetores e permite consultas eficientes por similaridade. Neste projeto, utilizamos o Qdrant \cite{qdrant_overview,qdrant_collections}.

\subsection{Proveniência e FAIR}

\textbf{Proveniência} é informação sobre origem e transformações do dado, útil para auditoria e reuso. O modelo PROV-DM do W3C formaliza esse conceito \cite{w3c_prov_dm}. Também buscamos aderência prática aos princípios \textbf{FAIR} (Encontrável, Acessível, Interoperável e Reutilizável) \cite{wilkinson_fair_2016}.

% ======================================================================
\section{Metodologia}

\subsection{Visão geral do pipeline}

A Figura~\ref{fig:pipeline} resume o pipeline: (E1) municípios, (E2) descoberta SAPL, (E3) extração de PLs, (E4) ementa para ação, (E5) indexação vetorial e (E6) indicadores e agrupamento.

\begin{figure*}[t]
\centering
\resizebox{\textwidth}{!}{%
\begin{tikzpicture}[node distance=0.9cm, font=\small]
\tikzstyle{box}=[draw, rounded corners, align=center, inner sep=6pt]
\node[box] (e1) {E1\\Municípios (IBGE)};
\node[box, right=0.6cm of e1] (e2) {E2\\Descoberta\\SAPL};
\node[box, right=0.6cm of e2] (e3) {E3\\Extração\\de PLs};
\node[box, right=0.6cm of e3] (e4) {E4\\Ementa $\rightarrow$ Ação};
\node[box, right=0.6cm of e4] (e5) {E5\\Embeddings\\+ Qdrant};
\node[box, right=0.6cm of e5] (e6) {E6\\Indicadores\\+ Agrupamento};
\draw[->] (e1) -- (e2);
\draw[->] (e2) -- (e3);
\draw[->] (e3) -- (e4);
\draw[->] (e4) -- (e5);
\draw[->] (e5) -- (e6);
\end{tikzpicture}
}%
\caption{Pipeline conceitual do Sonar Municipal.}
\label{fig:pipeline}
\end{figure*}

\subsection{Descoberta de instâncias SAPL e extração de Projetos de Lei}
\label{sec:sapl}

\subsubsection{Universo de busca e cobertura territorial}

O quadro amostral (lista de municípios) foi obtido via \textbf{API de localidades do IBGE} \cite{ibge_localidades_api}. Essa escolha reduz viés de seleção por listas manuais e permite replicação a partir da mesma fonte.

\subsubsection{Geração de candidatos e validação por evidência pública}

A descoberta foi implementada como \textbf{geração de candidatos + validação}. A geração usa \textbf{heurísticas} (regra prática baseada em padrões observáveis) para criar endereços prováveis de portais SAPL. A validação checa se o conteúdo e o comportamento de rotas públicas são compatíveis com o sistema.

\begin{algorithm}[H]
\caption{Descoberta e validação de instâncias SAPL (visão conceitual)}
\label{alg:descoberta}
\begin{algorithmic}[1]
\Require Lista de municípios $M$ (IBGE)
\Ensure Lista de instâncias SAPL validadas $S$
\State $S \gets \emptyset$
\For{cada município $m \in M$}
  \State $C \gets$ gerar candidatos de URL para $m$ (heurísticas)
  \For{cada candidato $c \in C$}
    \If{rota pública de consulta responde e conteúdo é compatível com SAPL}
      \State adicionar $c$ em $S$
      \State \textbf{break} \Comment{evita duplicidade para o mesmo município}
    \EndIf
  \EndFor
\EndFor
\State \Return $S$
\end{algorithmic}
\end{algorithm}

\subsubsection{Estrutura conceitual da API do SAPL}

Quando exposta, a API tende a seguir REST \cite{fielding_rest_2000}, com endpoints de coleções e itens. Em várias instâncias, a descrição OpenAPI e a interface Swagger UI podem estar disponíveis \cite{openapi_3_2,swagger_ui}.

\subsubsection{Extração de Projetos de Lei}

A extração foi desenhada para ser adaptável ao \textbf{vocabulário local}. O método consulta o catálogo de tipos de matéria e identifica quais rótulos correspondem a ``Projeto de Lei''. Em seguida, percorre todas as páginas até o esgotamento (paginação) \cite{drf_pagination}.

\begin{algorithm}[H]
\caption{Extração exaustiva de PLs via API (visão conceitual)}
\label{alg:extracao}
\begin{algorithmic}[1]
\Require Instância SAPL validada $s$
\Ensure Conjunto de PLs $P$ extraídos de $s$
\State $P \gets \emptyset$
\State $T \gets$ consultar catálogo de tipos de matéria em $s$
\State $T_{pl} \gets$ selecionar tipos que correspondem a ``Projeto de Lei''
\For{cada tipo $t \in T_{pl}$}
  \State $page \gets 1$
  \While{há resultados na página $page$}
    \State $R \gets$ requisitar lista paginada de matérias do tipo $t$ na página $page$
    \State $P \gets P \cup R$
    \State $page \gets page + 1$
  \EndWhile
\EndFor
\State \Return $P$
\end{algorithmic}
\end{algorithm}

\subsubsection{Resultado da primeira execução e customizações locais}

Na primeira execução dos coletores, foram encontradas 1.259 instâncias SAPL; porém, a extração automatizada de PLs foi possível em 322 instâncias, totalizando 220.065 PLs coletados. A discrepância é explicada por customizações locais (mudanças de rotas, proxies, autenticação, regras anti-robô e parametrizações).

\subsubsection{Rastreabilidade}

Para auditoria, cada registro mantém proveniência mínima: instância de origem, identificadores e carimbo temporal. O PROV-DM formaliza esse conceito \cite{w3c_prov_dm}.

% ======================================================================
\subsection{Construção do tradutor de ementas para ações}
\label{sec:tradutor}

Ementas têm variação alta e linguagem abreviada. Para reduzir ruído na busca e melhorar a leitura do resultado, o projeto traduz ementas para uma \textbf{recomendação de ação}, mantendo a semântica.

Foi criado um dataset de pares \textit{ementa $\rightarrow$ ação} com 1.000 amostras sintéticas, geradas por um LLM da família GPT-5.1 \cite{openai_gpt51_model,openai_gpt51_system_card}. O objetivo é transferência de aprendizado, isto é, aproveitar um modelo já treinado para aprender a tarefa com menos exemplos reais.

O modelo tradutor foi treinado como tarefa \textbf{seq2seq} (sequência para sequência). O ajuste foi feito com QLoRA-4bit \cite{dettmers_qlora_2023}. A qualidade foi avaliada por BERTScore \cite{zhang_bertscore_2019}, obtendo 84\% na avaliação interna do projeto.

Após o treinamento, todas as ementas coletadas foram convertidas em ações, formando um domínio fechado de ações recomendáveis.

% ======================================================================
\subsection{Codificação das ações para busca semântica}
\label{sec:busca}

Para indexar ações e consultas no mesmo espaço vetorial, foi utilizado o Multilingual E5 \cite{wang_e5_2024}, adequado ao cenário de pergunta e resposta. As embeddings foram armazenadas no Qdrant \cite{qdrant_overview,qdrant_collections}, permitindo recuperar as top-$K$ ações mais similares a uma pergunta.

Para viabilizar hospedagem gratuita do serviço de embeddings, uma opção prática é usar Text Embeddings Inference (TEI) \cite{hf_tei}.

% ======================================================================
\subsection{Indicadores para simulação de efeito no mundo real}
\label{sec:indicadores}

Indicadores em séries temporais foram construídos por município, usando fontes oficiais. No indicador educacional, por exemplo, utilizam-se microdados do Censo Escolar (INEP) \cite{inep_microdados_censo_escolar}. A comparação do indicador na data de apresentação do PL com valores meses ou anos depois produz uma estimativa operacional de variação, com horizonte escolhido pelo usuário.

\textbf{Observação:} essa estimativa é associativa, não causal. Ela descreve o que mudou depois, mas não prova que o PL causou a mudança.

% ======================================================================
\subsection{Agrupamento de projetos em políticas públicas}
\label{sec:agrupamento}

Para aumentar robustez, PLs similares são agrupados em clusters chamados ``políticas públicas''. A análise conjunta permite calcular efeito médio e dispersão. Para ranquear políticas, foi definida a métrica:

\[
Q = \frac{n_{\text{positivos}}}{n} \times \frac{n}{n+1}
\]

onde $n_{\text{positivos}}$ é o número de municípios com efeito considerado positivo (de acordo com o objetivo do indicador) e $n$ é o total de municípios no grupo.

% ======================================================================
\subsection{Desenvolvimento da plataforma web}
\label{sec:plataforma}

A plataforma foi desenvolvida em React com Next.js \cite{nextjs_docs} e hospedada na Vercel \cite{vercel_nextjs}. A arquitetura integra: interface web, serviço de embeddings e banco vetorial (Qdrant).

Funcionalidades principais:
\begin{enumerate}[leftmargin=*]
  \item recomendação de políticas públicas com estimativa de efeito esperado;
  \item detalhamento de PLs (ementa, data, link do SAPL e gráfico do indicador);
  \item busca semântica no acervo para apoiar ideação e comparação de propostas.
\end{enumerate}

% ======================================================================
\section{Resultados}

\begin{table}[H]
\centering
\caption{Resumo da primeira execução de descoberta e extração.}
\label{tab:coleta}
\begin{tabular}{l r}
\toprule
\textbf{Métrica} & \textbf{Valor} \\
\midrule
Instâncias SAPL encontradas & 1.259 \\
Instâncias com extração bem-sucedida & 322 \\
Projetos de Lei coletados (total) & 220.065 \\
\bottomrule
\end{tabular}
\end{table}

O tradutor ementa $\rightarrow$ ação atingiu BERTScore de 84\% na avaliação interna, indicando preservação semântica adequada para uso prático.

% ======================================================================
\section{Limitações e ameaças à validade}

\begin{itemize}[leftmargin=*,nosep]
  \item disponibilidade operacional de instâncias (instabilidade e mudanças de URL);
  \item heterogeneidade e customizações locais que quebram padrões de extração;
  \item variação semântica de rótulos e tipos de proposição entre instâncias;
  \item dificuldade de atribuir variações de indicadores a um único PL;
  \item uso de dataset sintético pode introduzir vieses do gerador.
\end{itemize}

% ======================================================================
\section{Conclusão}

O Sonar Municipal integra coleta sistemática de PLs via SAPL, tradução de ementas para ações, busca semântica por embeddings, análise com indicadores e agrupamento de políticas públicas. O resultado é uma ferramenta que aproxima evidência legislativa e dados oficiais do fluxo de trabalho de gestores públicos, com foco em reprodutibilidade e auditoria.

% ======================================================================
\printbibliography

\end{document}
